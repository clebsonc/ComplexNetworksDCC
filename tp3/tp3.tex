\documentclass[12pt, letterpaper]{article}
\usepackage[utf8]{inputenc}
\usepackage[letterpaper, margin=3.cm]{geometry}
\usepackage{graphicx}
\graphicspath{{/home/clebson/Documents/ComplexNetworksDCC/tp3/img//}}

\usepackage{epstopdf}
\epstopdfsetup{outdir=./}


\title{Experimento com NetLogo}
\author{Clebson C. A. de Sá}

\begin{document}
\maketitle

\section{Small World}
A primeira questão pede para considerar o modelo ``Small-World``.
A análise consistiu em uma rede baseada em 40 nós, no qual uma série de experimentos
foram realizados no software NetLogo com o intuíto de entender as características 
da rede.
A primeira observação a ser feita é que as conexões de cada nó levam em 
consideração a probabilidade de conexão entre os nós. No programa é possível
indicar a probabilidade que deseja avaliar por meio da variação do 
parâmetro ``rewiring-probability''.
As ligações são aleatoriamente proporcionais a esta probabilidade. Assim sendo,
para se ter uma boa estimativa deste parâmetro torna-se necessário efetuar repetições 
do experimento para computar o desvio padrão e entender melhor a variação em relação
a média. Para capturar esta informação foram executados 10 repetições para cada variação do parâmetro 
``rewiring-probability'' considerando o intervalo de $\left[0.1, 0.9\right]$ com
intercalação de $0.1$.
Os resultados para este experimento podem ser visualizados na Figura \ref{fig:small-world}
para ambas as métricas avaliadas com o devido desvio padrão.

\begin{figure*}[h]
  \centering
  \includegraphics[scale=0.4]{graphProb1.pdf}
  \caption{Coeficiente de Clusterização e Diâmetro com variação da probabilidade.}
  \label{fig:small-world}
\end{figure*}


Conforme podemos visualizar nesta figura, os valores para o Coeficiente de Clusterização
e Diâmetro diminuem conforme aumenta-se a probabilidade de linkagem entre os nós.
Isto é também confirmado ao observar o coeficiente de correlação Pearson de $0.88$
considerando todas as amostras do experimento conforme mostrado na Figura \ref{fig:correlation}.

\begin{figure*}[h]
  \centering
  \includegraphics[width=1\textwidth]{graphCorrelation.pdf}
  \caption{Correlação de Pearson com Regressão Linear das métricas avaliadas.}
  \label{fig:correlation}
\end{figure*}

Ainda nesta Figura, podemos observar a distribuição de ambas as métricas
por meio do histograma em ambos os eixos. O histograma do  Coeficiente de Clusterização nos mostra
que grande maioria dos valores observados estão espalhados entre 
$\left[0.05, 0.15\right]$. Comparando este intervalo da distribuição de 
clusterização podemos explicar melhor o motivo do alto valor do
desvio padrão na Figura \ref{fig:small-world} em torno da probabilidade
$\left[0.7, 0.9\right]$.
Em relação à distribuição do Diâmetro podemos observar no histograma que a maior
parte dos valores observados estão entre o intervalo $\left[2.5, 3.0\right]$.

Podemos concluir que estes resultados de fato fazem sentido, visto que batem com
o conceito de Redes de mundo pequeno, visto que com o aumento da probabilidade
existe o aumento da quantidade de links  entre os nós. Logo podemos inferir os valores para quaisquer
probabilidades por meio da regressão linear também mostrada na Figura \ref{fig:correlation}.



\section{Aids}
bla



\end{document}
