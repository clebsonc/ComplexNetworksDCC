\documentclass[12pt, letterpaper, twosided]{article}
\usepackage[utf8]{inputenc}

\title{Article: Strategic Network Formation with Structural Holes}
\author{Clebson C. A. de Sá}

\begin{document}
\maketitle


The paper approaches the importance of an agent that serves as intermediary bridges between groups of users. 
An interesting aspect of these agents is that they have power, and this power can be used for personal purposes, such as influencing the entire group that they are in. 
The author arguments that in a real life situation, these agents can be seem as managers within a large company.
The main discussion of the article is that in practice, there is a strategic aspect to the theory in which the network is shaped by agents who are trying to create bridging structures.
Thus, the paper main idea is to study the structure of the network when there are many agents trying to act as these local bridges.

To analyse these structure, the author defines a social network as an undirected graph, in which nodes are agents and the relationship of two agents is an edge.
The graph analysis basically consists of modelling the graph as a game. This game has the following properties: 
\textit{(i)} Every agent $u$ connected to $v$ has an cost $c(u, v)$ that defines the effort to maintain the relationship of the two agents. 
\textit{(ii)} Every node derives a direct benefit $\alpha_0$ from each of their nodes. 
\textit{(iii)} The importance for each node is basically cost matrix. The author defines two metrics of cost. The first one is a uniform cost, which considers that all nodes in the graph has a cost of $1$, and a hierarchical cost, that considers that the cost is the shortest paths in a tree representing a organizational structure.

TODO: Conclusion of paper

TODO: Relationship with metrics that we've study in class

\end{document}
